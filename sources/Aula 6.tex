\date{22-10-2024}

\section{Plano no $R^3$}

3 pontos não colineares no espaço só definem um plano. Seja $\pi$ um plano no espaço, $P=(x_0, y_0, z_0)$ um ponto em
$\pi$ e $\overrightarrow{n} = < a,b,c >$ um vetor ortogonal a $\pi$. Isto é, $\overrightarrow{n} * \overrightarrow{QR} = 0$, quaisquer que sejam $Q$ e $R$ em $\pi$

%grafico 

Se $Q = (x,y,z)$ em $\pi$ então $\overrightarrow{PQ} = < x-x_0, y-y_0, z-z_0 >$ é ortogonal a $\overrightarrow{n}$, isto é: 

\begin{equation}\label{vetorial}
    < x- x_0, y- y_0, z - z_0 > * <a,b,c> = 0 \leftrightarrow a(x-x_0) + b(y-y_0) + c (z-z_0) = 0
\end{equation}
\eqref{vetorial} equação vetorial de $\pi$

Podemos então reescrever \eqref{vetorial} como: 
\begin{equation}\label{geral}
    ax + by + xz = ax_0 + by_0 + cz_0 -> d \leftrightarrow ax + by + cz = d
\end{equation}
\eqref{geral} equação geral de $\pi$

\paragraph{Exemplo:}
Escreva a equação do plano que contém $P = (1,1,-2)$, $Q = (0,2,1)$ e $R = (-1,-1,0)$

%grafico

Para determinar um vetor ortogonal ao plano, basta tomar o produto vetorial $\overrightarrow{n} = \overrightarrow{PQ} * \overrightarrow{PR} = <-1,1,3> * <-2,-2,2>$

\[
\begin{vmatrix}
    \hat{i} & \hat{j} & \hat{k} \\ 
    -1 & 1 & 3 \\ 
    -2 & -2 & 2 
\end{vmatrix}
\] = $<8,-4,4>$

Daí, a equação vetorial do plano que contem P, Q e R é: $8(x-1) - 4(y-1) + 4(z+2) = 0$.
A equação geral é $8x-4y+4z = 8-4-8 = -4 \leftrightarrow 8x - 4y + 8z = -4$

\paragraph{Exemplo:} Somente a equação do plano que passa por $(1,4,3)$ e contém a reta:
\begin{equation}
    x = \frac{y-1}{2} = z+1 
\end{equation}

\paragraph{1º Solução:} Percebe que $P=(1,4,3)$, $Q=(0,1,-1)$ e $R = (1,3,0)$ pertecem ao plano $\pi$. 

\subsection{Distnacia entre um ponto e um plano}

Trocando po P uma reta paralela ao vetor normal, $(\overrightarrow{n})$ ao plano e denotando por R o ponto onde tal reta interecepta o plano $\pi$, definindo a distância de P em $\pi$ por:
$d = || \overrightarrow{RP} || = || P-R || $

Assim, 
$d = || Proj_{\overrightarrow{n} \overrightarrow{PQ}}$